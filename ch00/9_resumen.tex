
\chapter{Resumen}

La descripción automática del contenido de música grabada es crucial para la interacción con grandes colecciones de grabaciones de audio y para el desarrollo de nuevas herramientas que faciliten la pedagogía musical. La melodía es un aspecto fundamental para la mayoría de las tradiciones musicales, y es por tanto un componente indispensable para tal descripción. En esta tesis desarrollamos propuestas computacionales para el análisis de aspectos melódicos de alto nivel en interpretaciones musicales de Música Clásica de la India (MCI), con las que podemos describir e interrelacionar grandes cantidades de grabaciones de audio. Debido a su complejidad melódica y a su sólido marco teórico, la descripción de la melodía en MCI más allá de la línea melódica supone un interesante y desafiante objeto de investigación. Analizamos melodías en su contexto tonal, identificamos patrones melódicos, comparamos ambos tanto en piezas individuales como entre diferentes piezas, y finalmente caracterizamos el contexto melódico específico de MCI, los rāgas. Todos estos análisis se llevan a cabo mediante métodos dirigidos por datos en corpus de música de considerable tamaño y meticulosamente organizados.

La tesis comienza con la confección y estructuración de los mayores corpus musicales hasta la fecha de las dos tradiciones de MCI, indostaní y carnática. Dichos corpus están formados por grabaciones de audio de alta calidad y sus correspondientes metadatos. De estas extraemos la línea melódica predominante y la normalizamos según la tónica de su contexto. Un elemento importante para la descripción de melodías es la identificación de unidades temporales significativas, para lo que proponemos detectar en música indostaní las ocurrencias de nyās svaras, marcas que delimitan patrones melódicos musicalmente prominentes.

A partir de estas características melódicas, extraemos patrones melódicos recurrentes y musicalmente relevantes. Estos patrones son las unidades básicas con las que se construyen estructuras melódicas tanto en improvisaciones como composiciones, y por tanto son fundamentales para la descripción de colecciones de audio en MCI. Proponemos un método no supervisado basado en el análisis de las series temporales para el descubrimiento de patrones melódicos en colecciones musicales de tamaño considerable. En primer lugar llevamos a cabo un análisis supervisado en profundidad de similitud melódica, que es el componente crítico para el descubrimiento de patrones. A continuación mejoramos la propuesta más competitiva aprovechando las características melódicas propias de MCI. Para identificar patrones musicalmente significativos, aprovechamos las relaciones entre los patrones descubiertos mediante la implementación de análisis de redes. Exhaustivas evaluaciones auditivas por parte de músicos profesionales de los patrones melódicos descubiertos revelan que estos son musicalmente interesantes y significativos.

Finalmente, utilizamos nuestros resultados para el reconocimiento de rāgas en interpretaciones grabadas de MCI. Proponemos dos métodos nuevos que captan conjuntamente los aspectos tonales y temporales de la melodía. Nuestro primer método se sirve de patrones melódicos, los principales indicadores para la identificación de rāgas por parte de oyentes humanos. Utilizamos los patrones melódicos descubiertos y empleamos técnicas de modelado de temas, en las que equiparamos la interpretación de un rāga a la descripción textual de un tema. En nuestro segundo método, proponemos una superficie melódica de tiempo de retardo, una característica nueva basada en las coordinadas de retraso que captan el contorno melódico de un rāga. Con estos métodos alcanzamos precisiones sin precedentes en el reconocimiento de rāgas en los mayores conjuntos de datos nunca usados para esta tarea. Aunque nuestra propuesta se fundamenta en las características de las melodías en MCI y la tarea en cuestión, creemos que nuestra metodología puede ser fácilmente aplicable a otras tradiciones musicales predominantemente melódicas.

En general, hemos construido nuevos métodos computacionales para el análisis de varios aspectos melódicos de interpretaciones grabadas de MCI, con las que describimos e interrelacionamos grandes cantidades de grabaciones musicales. En este proceso hemos desarrollado varias herramientas y reunido datos que pueden ser empleados en numerosos estudios computacionales de MCI, específicamente para la caracterización de rāgas, composiciones y artistas. Las tecnologías resultantes de este trabajo de investigación son parte de varias aplicaciones desarrolladas en el proyecto CompMusic para la mejora de la descripción, experiencia de escucha, y enseñanza de MCI.

\vfill
{\small \noindent (\emph{Translated from English by Rafael Caro Repetto})}