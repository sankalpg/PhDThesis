
Automatically describing different aspects of digital music content is crucial for organizing, searching and interrelating large volumes of music recordings. Melody is a fundamental facet in most music traditions, and therefore, is an indispensable component in this task. It is still a challenge  to obtain a reliable melody representation from recorded polyphonic music and most melodic analyses primarily use symbolic representations of music, thus covering only a particular view of music. The specific heterophonic characteristics in \gls{iam} make it feasible to obtain a low-level melody representation from audio recordings using current state of the art pitch estimation methods, which in turn enables to focus on the description of higher level melodic aspects of music performances. With its complex melodic framework and well-grounded theory, the description of \gls{iam} melody  beyond pitch contours, offers a very interesting and challenging research topic. In this thesis we analyze melodies within their tonal context, identify melodic patterns and compare them both within and across music pieces, and finally, characterize the specific melodic context of \gls{iam}, the \glspl{raga}. All these analyses are done using data-driven methodologies on large curated music corpora. 

The thesis starts by compiling and structuring representative music corpora of the two \gls{iam} traditions, Hindustani and Carnatic music, comprising quality audio recordings and the associated metadata. From the music recordings we extract low-level melody representations by using a state of the art algorithm for extracting predominant pitch from polyphonic audio. To make this representation more musically meaningful we identify the tonic context of every recording, estimating the  tonic by using a multi-pitch approach, which is then used to normalize the pitch contours. Another important element to describe melodies is the identification of the meaningful temporal units, for which we propose to detect occurrences of \gls{nyas} \glspl{svara} in Hindustani music, a landmark that demarcates musically salient melodic patterns.

Utilizing the melodic features mentioned above we extract musically significant melodic patterns. These recurring patterns are the building blocks of melodic structures in both improvisation and composition, and are thus, fundamental to the description of audio collections in \gls{iam}. We propose an unsupervised approach that employs time-series analysis tools to discover melodic patterns in sizable music collections. Due to the challenges involved in quantitatively evaluating unsupervised approaches, we first carry out an in-depth supervised analysis of  melodic similarity, which is a critical component in pattern discovery. We improve upon the best methodology for melodic similarity that we identified through an exhaustive evaluation of different procedures and parameter settings that are well known and widely used for this task. We improve melodic similarity by exploiting characteristics that are peculiar to melodies in \gls{iam}. Computational pattern discovery often results in large quantities of musically irrelevant patterns. To identify and select musically significant patterns we exploit the relationships between the discovered patterns by performing a network analysis. Listening tests reveal that the discovered melodic patterns are musically interesting and significant.

Finally, we utilize the results of the analysis  mentioned above - melody representations, melodic descriptors and discovered patterns for the recognition of \glspl{raga} in recorded performances of \gls{iam}. \Gls{raga} is a core musical concept in \gls{iam}, used in composition, performance, music organization, and pedagogy, and therefore, the most desired melodic description of a music recording. We propose two novel approaches that jointly capture the tonal and the temporal aspects of melody. Our first approach uses melodic patterns, the most prominent cues for raga identification by humans. We utilize the discovered melodic patterns and employ topic modeling techniques, wherein we regard a \gls{raga} rendition similar to a textual description of a topic. In our second approach we propose the \gls{tdms}, a novel feature based on delay coordinates that captures the melodic outline of a \gls{raga}. With these approaches we demonstrate unprecedented accuracies in \gls{raga} recognition on the largest datasets ever used for this task.  Although our approach is guided by the characteristics of melodies in \gls{iam} and the task at hand, we believe our methodology can be easily extended to other melody dominant music traditions.

In this thesis we have built novel computational methods for melodic analysis of \gls{iam}, using which we can describe and interlink large amount of music recordings. In this process we have developed several tools and compiled data that can be used for a number of computational studies in \gls{iam}, specifically in characterization of ragas, compositions and artists. The technologies resulted from this research work are a part of several applications developed within the CompMusic project for a better description, enhanced listening experience, and pedagogy in \gls{iam}.

\newpage
\TODO{read below the comments that should be incorporated in the text.}
%These are the changes which are to be done based on the things learned in the abstract
%
%\begin{enumerate}
%	\item What is melody for us is not clear, definition wise. But Xavier's point is dont write that we represent melody using pitch. Because if that's the melody for us, then what are we doing? we are also in a way devising new melodic representations at higher-level. So say that we start from a low-level melody representation which is the pitch. Then we build upon that etc etc and perform melodic analyses. This change will go in 1) Melody definition big time! 2) Melody extraction in pre-pro 3) In every study you can replace melody representation with predominant pitch estimation. There are many changes for this point to be correctly written. 
%	
%	\item Justification of IAM: Rather than stressing on the culture specificity, put emphasis on the fact that we are able to get a decent pitch, as a result of which we can analyze melodies now which we couldn't do for other music traditions. So in audio melodic description is basically extracting pitch, now we go beyond that and address musical issues. And IAM makes it possible due to its specific heterophonic characteristics. 
%	
%	\item There is not a lot of need to justify why working on melody, its probably clear, just small para and that's it. 
%	
%	\item from the abstract some keywords can get into the intros of every concept in the methodology chapters. Carefully write, melody, pitch, method, algorithm, approach, phrase, pattern etc...
%	
%	\item when we are giving a summary or conclusion or a broad perspective consider highlighting the entire methodology that we start with a low level melody representation and generate a higher level description of it...just dont write we did melodic similarity and discovery blah blah
%	
%	\item Highlight the big data aspect of the study, and that the patterns are analyzed at the level of the corpus and not just a recording. 
%	
%	\item Organization word in the context of raga usage is not clear, change it to music organization everywhere.
%	
%	\item instead of saying representative of real world collections, you say representative of the music tradition. 
%	
%\end{enumerate}

%
%%%%%%%%%%%%%%Applications of the technologies we work on: Organization, Navigation, Recommendation, Discovery, Creation, Education, Evaluation, Enhancements (listening, performance), Musicology, Other studies
%
%%%%%% What actions can be done automatically to give rise to applications above
% ()Keywords) automatic description, indexing, search, retrieval, interaction with music content 
%
%%%%%%%%%%%%%Applications/motivations of our specific tasks
%
%Tonic Identification: understanding musical concepts (drone + raga-tonic dualities etc), automatic music description, input to higher level analysis
%
%Nyas segmentation: Understanding musical concept, automatic description, input to other analyses, enhanced listening, education tool
%
%Similarity: understanding musical concept, automatic description, input to other analyses, enhanced listening, music education tools, establishing similarities and influences between artists, school of music, ragas, and recordings.... input to higher level analysis, navigation and discovery, musicological analysis, similarity based retrieval
%
%discovery: understanding musical concepts, understanding improvisation in \gls{iam}, understanding creative aspects of \gls{iam}, music generation, enhanced listening, music education tools, establishing similarities and influences between artists, school of music, ragas and recordings...input to higher level analysis, navigation and discovery, musicologial analysis, indexing, search and retrieval, similarity based retrieval
%
%Characterization: al that discovery does + understanding function roles of differnt type of patterns, 
%
%Raga recognition: understaing music, automatic description, education tool, establishing sim and inf like said above, raga based music retrieval, organization, navigation and discovery, indexing, searhc and retrieval 

%%%%%%%%%%%%%%% Different scientific areas that we use the concepts from
% Signal processing, time-series analysis, machine learning, musicology

%%%%%%%%%%%%%% Our approaches/work (adjectives) 
% Data-driven, involve both top-down and bottom-up approach, Culturally aware, human interpretable stages and results and learnings, applied research, focus on understanding than numbers, quantitative and qualitative evaluations, Knowledge driven??, domain-specific, content-based, mostly using audio signal information, 

%%%% Main goals
% Select relevant (core) problems, gather representative data, understand choices of parameters and processing steps since characteristics are so diff, understanding of challenges and opportunities in diff tasks, influence of characteistics on choices of procedure and params, influence of params and procedures on tasks, comparative eval whereever possible, compile literature, Evaluate and verify hypothesis and approaches, exploiting cultural specificities to advance!!! identify key melodic unit to describe melodies..exploiting specificities of musical culture..melodic pattern similarity, discovery and characterization, unsupervised analysis, bringing novel insights, large-scale study,  raga-recognition, quantitative evaluation of musical relevance of discovered patterns, novel melodic representation, pushing state of the art in raga recognition..taking analogies with topic modeling techniques, connection between describing a topic and rendering a raga. open data, open code, reproducibility, applications, demonstrating usefulness and potential of such technologies, 


