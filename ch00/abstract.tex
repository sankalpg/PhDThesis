
Automatically describing different aspects of digital music content is nowadays crucial for organizing, searching and interrelating large volumes of diverse music collections. Melody is a fundamental dimension in most music traditions, and therefore, is an indispensable component in their description. However, being a socio-cultural product, melody necessitates culture-aware methodologies for its analysis. This thesis addresses computational analysis of melodies in Indian art music to automatically describe and characterize its melodic structures directly from audio music recordings. Melodic analysis in Indian art music is largely underexplored by the current computational methodologies in music information research. With its complex melodic structures and well-grounded theory this music tradition provides an opportunity to push the boundaries of the current knowledge in MIR. 

To this end, we start by building a sizable data corpora that is representative of the real-world music collections of Indian art music, with which we then develop our data-driven approaches. We provide a detailed documentation of this process, and also elaborate different test datasets that we use in our work. In Indian art music a lead artist in a performance is free to choose any arbitrary tonic frequency. Therefore, identifying the tonic pitch used in a recording is critical for any melodic analysis that is performed across artists and their recordings. We perform a comprehensive comparative evaluation of the existing tonic identification approaches to identify the most reliable method to be used in our work. We propose an approach for detecting nyas landmarks in melodies, which facilitate segmentation of stable melodic regions and melodic patterns. Subsequently, we focus on the main tasks addressed in this thesis, melodic pattern discovery and raga recognition. 

Repeating melodic pattens are fundamental units in melodies of IAM, and therefore, they become the central XXX utilized for analyzing melodies in this study. 

Pattern discovery aims to extract musically significant melodic patterns in large audio music collections of Indian art music. We follow an unsupervised methodology and employ time-series analysis tools to discover repeating melodic patterns in sizable audio music collections. Since an quantitative evolution of such an unsupervised system is a challenging task, we study melodic similarity, a crucial component of the discovery system in an supervised setup. We study in-depth the influence of different computational procedures and parameter settings often used in this task on the computation of melodic similarity in Indian art music. We obtain the best set of procedures and parameter settings, and show that this task is very sensitive to these choices. In addition, we propose a novel approach that improves melodic similarity computation by exploiting specific melodic characteristics in Indian art music. Pattern discovery often results in a large fraction of musically trivial patterns. In order to identify the patterns that are musically significant we propose a novel approach that employs complex network analysis tools and exploits the functional roles of different melodic patterns in IAM to perform the task. Using listening tests we show that the discovered melodic patterns are musically interesting and significant.

Raga is a core musical concepts in Indian art music used in composition, performance, music organization, and pedagogy. Automatically identifying raga in an audio recording is one of the most relevant topics in computational analysis of IAM. We propose two novel approaches for raga recognition that jointly model the tonal and the temporal aspects of melody. Our first approach is based on melodic patterns, the most prominent cues for raga recognition. For this, we utilize discovery melodic patterns and employ topic modeling technique. We show a comparable accuracy to the state of the art using this approach. Our second approach proposes a novel melody representation that captures both the tonal and the temporal melodic aspects relevant in characterizing ragas. With this approach we show unprecedented accuracies in raga recognition task on sizable datasets. 

The technologies resulting from this research work are a part of a number of applications developed within the CompMusic project for a XXX of Indian art music. These applications capitalize on the automated description to provide enriched listening experience and discovery of music. The data and tools developed in this work are relevant to a number of musicological work, specifically in studying pattern based characterization of ragas, compositions and artists. There can be a number of MIR tasks that can build upon the discovered melodic patterns such as music recommendation, music similarity, and music indexing?



Technologies that can `make sense' of digital information such as digital music is now a days a necessity in modern world. Such technologies can be used to develop novel ways to interact with massively growing music content and to build innovative tools for music education and music creation. In most musics of the world melody is a fundamental dimension of music and its computational analysis is a core XXX for XXX. While there are several transversal aspects of melody across music traditions, there are significantly varied culture specific aspects that pose unique challenges to existing MIR approaches and call for a novel methodologies to address them. Focusing on such specificities can expand the existing knowledge and make MIR technologies more complete and versatile. 

In this thesis we aim to develop computational approaches for automatic description and analysis of melodies in \gls{iam}. Our data-driven approaches are based on signal processing, time-series analysis and machine learning concepts. We start with identifying relevant research problems that address the challenges, and benefit from the opportunities offered by the complex melodic structures in \gls{iam}. To ensure real-world scalability of our approaches, building representative data corpora and datasets has been one of the focus throughout the thesis. Recurring pattens are fundamental units in melodies of IAM, and therefore, they become the central XXX utilized for analyzing melodies in this study. Different approaches addressing relevant pattern processing tasks such as melodic similarity, patten discovery and pattern characterization are proposed. Our quantitative and qualitative evaluations show promising results. The music recordings, performances, pedagogy in IAM is structured around the concept of raga. We address the fundamental task of automatic raga recognition and propose two approaches. Our first approach utilize discovered melodic patterns for recognition ragas, wherein we borrow the concepts from topic modeling. Our second approach is based on a novel melodic representation that captures both tonal and temporal characteristics of melody. With our extensive evaluations and comparisons with the existing methods we show that we advance the state of the art in raga recognition. Data pre-processing? We presenting a number of applications built on top of our approaches that demonstrate the usefulness and relevance of the output of our approaches. 



%
%%%%%%%%%%%%%%Applications of the technologies we work on: Organization, Navigation, Recommendation, Discovery, Creation, Education, Evaluation, Enhancements (listening, performance), Musicology, Other studies
%
%%%%%% What actions can be done automatically to give rise to applications above
% ()Keywords) automatic description, indexing, search, retrieval, interaction with music content 
%
%%%%%%%%%%%%%Applications/motivations of our specific tasks
%
%Tonic Identification: understanding musical concepts (drone + raga-tonic dualities etc), automatic music description, input to higher level analysis
%
%Nyas segmentation: Understanding musical concept, automatic description, input to other analyses, enhanced listening, education tool
%
%Similarity: understanding musical concept, automatic description, input to other analyses, enhanced listening, music education tools, establishing similarities and influences between artists, school of music, ragas, and recordings.... input to higher level analysis, navigation and discovery, musicological analysis, similarity based retrieval
%
%discovery: understanding musical concepts, understanding improvisation in IAM, understanding creative aspects of IAM, music generation, enhanced listening, music education tools, establishing similarities and influences between artists, school of music, ragas and recordings...input to higher level analysis, navigation and discovery, musicologial analysis, indexing, search and retrieval, similarity based retrieval
%
%Characterization: al that discovery does + understanding function roles of differnt type of patterns, 
%
%Raga recognition: understaing music, automatic description, education tool, establishing sim and inf like said above, raga based music retrieval, organization, navigation and discovery, indexing, searhc and retrieval 

%%%%%%%%%%%%%%% Different scientific areas that we use the concepts from
% Signal processing, time-series analysis, machine learning, musicology

%%%%%%%%%%%%%% Our approaches/work (adjectives) 
% Data-driven, involve both top-down and bottom-up approach, Culturally aware, human interpretable stages and results and learnings, applied research, focus on understanding than numbers, quantitative and qualitative evaluations, Knowledge driven??, domain-specific, content-based, mostly using audio signal information, 

%%%% Main goals
% Select relevant (core) problems, gather representative data, understand choices of parameters and processing steps since characteristics are so diff, understanding of challenges and opportunities in diff tasks, influence of characteistics on choices of procedure and params, influence of params and procedures on tasks, comparative eval whereever possible, compile literature, Evaluate and verify hypothesis and approaches, exploiting cultural specificities to advance!!! identify key melodic unit to describe melodies..exploiting specificities of musical culture..melodic pattern similarity, discovery and characterization, unsupervised analysis, bringing novel insights, large-scale study,  raga-recognition, quantitative evaluation of musical relevance of discovered patterns, novel melodic representation, pushing state of the art in raga recognition..taking analogies with topic modeling techniques, connection between describing a topic and rendering a raga. open data, open code, reproducibility, applications, demonstrating usefulness and potential of such technologies, 


